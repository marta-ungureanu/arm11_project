\documentclass{article}
\usepackage[english]{babel}
\usepackage[a4paper, total={6in, 8in}]{geometry}

%%%%%%%%%% Start TeXmacs macros
\newcommand{\tmtextbf}[1]{{\bfseries{#1}}}
%%%%%%%%%% End TeXmacs macros

\begin{document}

\title{ARM Checkpoint}
\author{Group 3: Andrew Pearcy, Marta Ungureanu, Maurizio Zen, Oana Ciocioman}
\maketitle



\section{Group Organisation}
\begin{paragraph}
\tab
\end{paragraph}

\tmtextbf{How we split the work:}

Initially, we worked together to get the first common parts of the
functionality working. The Check Condition function and the main decode
function were written together so that we could then each implement one of the
four main functions. Once we had this initially setup working we designated
the four main tasks as follows. Andrew was to code the Data Processing
instructions, Marta the multiply instruction, Oana the Single Data Transfer
Instruction and Maurizio the Branch instruction.\newline 

We then communicated heavily over social media in order to ensure that our
independent parts would synchronise upon merging. Once we believed the main
functionality of all of our independent parts was working, we merged the code
together in order to run the more complicated tests, for example those which
used a branch instruction required the CPSR flags to be set, which is dealt
with under the Data Processing commands.\newline


\tmtextbf{How well is the group working:} 

Once we had our first team meeting and set out our goals on what each
individual should be trying to accomplish, we were very successful in
implementing our individual parts in a timely fashion. Upon merging all of the
work together we discovered a few bugs and worked together as a team to
identify exactly where the origin of these bugs were. It was necessary to go
through the bugs together because each person had a strong understanding of
their code. This allowed us to quickly run through each of the operations that
was being executed and work together to identify what the root of the cause
was and whether or not we had missed an edge case.\newline


For future improvement we will need to be more careful to consider test cases. 
In out first implementaiton required a lot of debugging as we had not taken into
account all the possible edge cases. Therefore we will need to be more meticulous
to consider all possibilities for the next parts of the project. \newline



\newpage 
\section{Implementation Strategies}
\begin{paragraph}
\tab
\end{paragraph}

\tmtextbf{Emulator Structure:}

Our emulator is structured as follows:

Initially, the contents of the file is read into our virtually memory which is
stored as a global struct along with our registers. Then the pipeline function
is called which simulates the pipeline as described in the spec. This function
uses the Program Counter (stored in the struct) to fetch the instruction from
memory. The little endianness of the memory is handled by the fetch instruction which
takes the independent bytes and returns a 32 bit instruction. This is then
stored in the pipeline as the fetched instruction. Since the pipeline's main
functionality is the while loop which causes it to continue execution, it
re-enters the loop and decodes the current instruction, fetching the next. The
decoded instruction is decoded using a separate decode instruction which
determines which of the four types of instruction to call. Hence the decoded
instruction is stored as a number between -1 and 4 with -1 representing an instruction not 
contained in our set, 0 being the Stop instruction, 1 being
multiply, 2 data processing, 3 single data transfer and 4 branch.\newline

On the next iteration of the loop, the pipeline continues its fetch decode
cycle, and now that is had a decoded instruction it executes this instruction
by calling execute on the decoded instruction which then determines which of
the four tasks should be run. The four tasks are then passed the instruction
to decode the necessary bits remaining in the instruction to determine exactly
what they need to do. For example with data processing it needs to determine
the opcode in order to carry out the right instruction. The multiply
instruction determines which registers to operates on, the single data
transfer has to determine which registers and what type of operation, and the
branch instruction has to determine the offset.\newline

After the instruction is executed, the pipeline continues to run until it
reaches a 0 instruction at which point it exits and prints the contents of the
registers and the memory with our print status function.\newline

In terms of what parts we will be able to reuse for our
assembler, we believe it is unlikely that much of the code will be able to
re-used, except in a reverse engineering aspect. Since our emulator deals with
decoding the binary files, we should be able to use this as a basis/target of
what our assembler will be aiming to produce. We will be able to use our
emulator to test our assembled files, but beyond that the emulator will be
somewhat limited in its usefulness for implementing the assembler, except for
perhaps the function which reads into memory, a similar function will be
required for assembler. Additionally, we should be able to reuse our structure 
which stores the current state of our machine, i.e. the current register values
and memory values. By having this in a separate file we will be able to include
it in the assembler.\newline



\tmtextbf{What implementation tasks we may find difficult:}

We think one of the most difficult part of assembler will be recognising
labels within the assembly code. We will have to make a decision about how we
want to implement this, whether by a one pass or two pass system, but
recognising these and assigning them appropriately for branch statements will
definitely be a challenge.

\end{document}
